\documentclass[onecolumn,twoside,11pt]{article}

\usepackage{CJK}
\usepackage{graphicx}
\usepackage{amssymb}
\usepackage{bm}
\usepackage{color}
\usepackage{fancyhdr}
\pagestyle{fancy}

%%%%%%%%%%%%%%%%%%%%%%%%%%%%%%%%%%%%%%%%%%%%%%%%%%%%%%%%%%%%%%%%%%%%%%%%%%%
%                             P A G E   S I Z E
%%%%%%%%%%%%%%%%%%%%%%%%%%%%%%%%%%%%%%%%%%%%%%%%%%%%%%%%%%%%%%%%%%%%%%%%%%%

\renewcommand{\topfraction}{0.95}   % let figure take up nearly whole page
\renewcommand{\textfraction}{0.05}  % let figure take up nearly whole page

% Specify the dimensions of each page

\oddsidemargin .25in    %   Note \oddsidemargin = \evensidemargin
\evensidemargin .25in
\marginparwidth 0.07 true in
%\marginparwidth 0.75 true in
%\topmargin 0 true pt           % Nominal distance from top of page to top of
%\topmargin 0.125in
\topmargin -0.5in
\addtolength{\headsep}{0.25in}
\textheight 8.5 true in       % Height of text (including footnotes & figures)
\textwidth 6.0 true in        % Width of text line.
\widowpenalty=10000
\clubpenalty=10000

%%%%%%%%%%%%%%%%%%%%%%%%%%%%%%%%%%%%%%%%%%%%%%%%%%%%%%%%%%%%%%%%%%%%%%%%%%%
%                               S E C T I O N S
%%%%%%%%%%%%%%%%%%%%%%%%%%%%%%%%%%%%%%%%%%%%%%%%%%%%%%%%%%%%%%%%%%%%%%%%%%%

\makeatletter
\def\@startsiction#1#2#3#4#5#6{\if@noskipsec \leavevmode \fi
   \par \@tempskipa #4\relax
   \@afterindenttrue
   \ifdim \@tempskipa <\z@ \@tempskipa -\@tempskipa \@afterindentfalse\fi
   \if@nobreak \everypar{}\else
     \addpenalty{\@secpenalty}\addvspace{\@tempskipa}\fi \@ifstar
     {\@ssect{#3}{#4}{#5}{#6}}{\@dblarg{\@sict{#1}{#2}{#3}{#4}{#5}{#6}}}}

\def\@sict#1#2#3#4#5#6[#7]#8{\ifnum #2>\c@secnumdepth
     \def\@svsec{}\else
     \refstepcounter{#1}\edef\@svsec{\csname the#1\endcsname}\fi
     \@tempskipa #5\relax
      \ifdim \@tempskipa>\z@
        \begingroup #6\relax
          \@hangfrom{\hskip #3\relax\@svsec.\hskip 0.1em}
                    {\interlinepenalty \@M #8\par}
        \endgroup
       \csname #1mark\endcsname{#7}\addcontentsline
         {toc}{#1}{\ifnum #2>\c@secnumdepth \else
                      \protect\numberline{\csname the#1\endcsname}\fi
                    #7}\else
        \def\@svsechd{#6\hskip #3\@svsec #8\csname #1mark\endcsname
                      {#7}\addcontentsline
                           {toc}{#1}{\ifnum #2>\c@secnumdepth \else
                             \protect\numberline{\csname the#1\endcsname}\fi
                       #7}}\fi
     \@xsect{#5}}

\def\@sect#1#2#3#4#5#6[#7]#8{\ifnum #2>\c@secnumdepth
     \def\@svsec{}\else
     \refstepcounter{#1}\edef\@svsec{\csname the#1\endcsname\hskip 0.5em }\fi
     \@tempskipa #5\relax
      \ifdim \@tempskipa>\z@
        \begingroup #6\relax
          \@hangfrom{\hskip #3\relax\@svsec}{\interlinepenalty \@M #8\par}
        \endgroup
       \csname #1mark\endcsname{#7}\addcontentsline
         {toc}{#1}{\ifnum #2>\c@secnumdepth \else
                      \protect\numberline{\csname the#1\endcsname}\fi
                    #7}\else
        \def\@svsechd{#6\hskip #3\@svsec #8\csname #1mark\endcsname
                      {#7}\addcontentsline
                           {toc}{#1}{\ifnum #2>\c@secnumdepth \else
                             \protect\numberline{\csname the#1\endcsname}\fi
                       #7}}\fi
     \@xsect{#5}}

\def\thesection {\arabic{section}}
\def\thesubsection {\thesection.\arabic{subsection}}
\renewcommand\section{\@startsiction{section}{1}{\z@}{-0.24in}{0.10in}
             {\large\bf\raggedright}}
\renewcommand\subsection{\@startsection{subsection}{2}{\z@}{-0.20in}{0.08in}
                {\normalsize\bf\raggedright}}
\renewcommand\subsubsection{\@startsection{subsubsection}{3}{\z@}{-0.18in}{0.08in}
                {\normalsize\sc\raggedright}}
\renewcommand\paragraph{\@startsiction{paragraph}{4}{\z@}{1.5ex plus
  0.5ex minus .2ex}{-1em}{\normalsize\bf}}
\renewcommand\subparagraph{\@startsiction{subparagraph}{5}{\z@}{1.5ex plus
  0.5ex minus .2ex}{-1em}{\normalsize\bf}}
\makeatother

%%%%%%%%%%%%%%%%%%%%%%%%%%%%%%%%%%%%%%%%%%%%%%%%%%%%%%%%%%%%%%%%%%%%%%%%%%%
%                               A B S T R A C T
%%%%%%%%%%%%%%%%%%%%%%%%%%%%%%%%%%%%%%%%%%%%%%%%%%%%%%%%%%%%%%%%%%%%%%%%%%%

\renewenvironment{abstract}
{\centerline{\large\bf Abstract}\vspace{0.7ex}%
  \bgroup\leftskip 20pt\rightskip 20pt\small\noindent}%
{\par\egroup\vskip 0.25ex}

%%%%%%%%%%%%%%%%%%%%%%%%%%%%%%%%%%%%%%%%%%%%%%%%%%%%%%%%%%%%%%%%%%%%%%%%%%%
%                         KEYWORDS THEOREM ACK NOTES
%%%%%%%%%%%%%%%%%%%%%%%%%%%%%%%%%%%%%%%%%%%%%%%%%%%%%%%%%%%%%%%%%%%%%%%%%%%

\newenvironment{keywords}
{\bgroup\leftskip 20pt\rightskip 20pt \small\noindent{\bf Keywords:} }%
{\par\egroup\vskip 0.25ex}

\newcommand{\BlackBox}{\rule{1.5ex}{1.5ex}}  % end of proof
\newenvironment{proof}{\par\noindent{\bf Proof\ }}{\hfill\BlackBox\\[2mm]}
\newtheorem{example}{Example}
\newtheorem{theorem}{Theorem}
\newtheorem{lemma}[theorem]{Lemma}
\newtheorem{proposition}[theorem]{Proposition}
\newtheorem{remark}[theorem]{Remark}
\newtheorem{corollary}[theorem]{Corollary}
\newtheorem{definition}[theorem]{Definition}
\newtheorem{conjecture}[theorem]{Conjecture}
\newtheorem{axiom}[theorem]{Axiom}

\long\def\acks#1{\vskip 0.3in\noindent{\large\bf Acknowledgments}\vskip 0.2in\noindent #1}
\long\def\researchnote#1{\noindent {\LARGE\it Research Note} #1}

%%%%%%%%%%%%%%%%%%%%%%%%%%%%%%%%%%%%%%%%%%%%%%%%%%%%%%%%%%%%%%%%%%%%%%%%%%%
%                        FIRST PAGE, TITLE, AUTHOR
%%%%%%%%%%%%%%%%%%%%%%%%%%%%%%%%%%%%%%%%%%%%%%%%%%%%%%%%%%%%%%%%%%%%%%%%%%%

\newlength\aftertitskip     \newlength\beforetitskip
\newlength\interauthorskip  \newlength\aftermaketitskip

%% Changeable parameters.
\setlength\aftertitskip{0.1in plus 0.2in minus 0.2in}
\setlength\beforetitskip{0.05in plus 0.08in minus 0.08in}
\setlength\interauthorskip{0.08in plus 0.1in minus 0.1in}
\setlength\aftermaketitskip{0.3in plus 0.1in minus 0.1in}

%=================================================================
%%CONTENTS
%=================================================================

% Definitions of handy macros
\newcommand{\fracpartial}[2]{\frac{\partial #1}{\partial  #2}}

\begin{document}
\begin{CJK*}{GBK}{song}

\title{Data Mining Project: Prediction of Tokyo Water Supply}
\fancyhf{}
\fancyhead[CE]{Roi ZHAO}
\fancyfoot[RO,LE]{\thepage}
\fancyfoot[LO]{NEC Labs China}
\fancyheadoffset{4cm}

%\maketitle
{ \huge \bfseries \centering iconv: illegal input sequence at position 6574
tex iconv -i gbk -t utf-8 sample.tex 
iconv: invalid option -- 'i'
Try `iconv --help' or `iconv --usage' for more information.
tex iconv -f gbk -t utf-8 sample.tex 
\documentclass[onecolumn,twoside,11pt]{article}

\usepackage{CJK}
\usepackage{graphicx}
\usepackage{amssymb}
\usepackage{bm}
\usepackage{color}
\usepackage{fancyhdr}
\pagestyle{fancy}

%%%%%%%%%%%%%%%%%%%%%%%%%%%%%%%%%%%%%%%%%%%%%%%%%%%%%%%%%%%%%%%%%%%%%%%%%%%
%                             P A G E   S I Z E
%%%%%%%%%%%%%%%%%%%%%%%%%%%%%%%%%%%%%%%%%%%%%%%%%%%%%%%%%%%%%%%%%%%%%%%%%%%

\renewcommand{\topfraction}{0.95}   % let figure take up nearly whole page
\renewcommand{\textfraction}{0.05}  % let figure take up nearly whole page

% Specify the dimensions of each page

\oddsidemargin .25in    %   Note \oddsidemargin = \evensidemargin
\evensidemargin .25in
\marginparwidth 0.07 true in
%\marginparwidth 0.75 true in
%\topmargin 0 true pt           % Nominal distance from top of page to top of
%\topmargin 0.125in
\topmargin -0.5in
\addtolength{\headsep}{0.25in}
\textheight 8.5 true in       % Height of text (including footnotes & figures)
\textwidth 6.0 true in        % Width of text line.
\widowpenalty=10000
\clubpenalty=10000

%%%%%%%%%%%%%%%%%%%%%%%%%%%%%%%%%%%%%%%%%%%%%%%%%%%%%%%%%%%%%%%%%%%%%%%%%%%
%                               S E C T I O N S
%%%%%%%%%%%%%%%%%%%%%%%%%%%%%%%%%%%%%%%%%%%%%%%%%%%%%%%%%%%%%%%%%%%%%%%%%%%

\makeatletter
\def\@startsiction#1#2#3#4#5#6{\if@noskipsec \leavevmode \fi
   \par \@tempskipa #4\relax
   \@afterindenttrue
   \ifdim \@tempskipa <\z@ \@tempskipa -\@tempskipa \@afterindentfalse\fi
   \if@nobreak \everypar{}\else
     \addpenalty{\@secpenalty}\addvspace{\@tempskipa}\fi \@ifstar
     {\@ssect{#3}{#4}{#5}{#6}}{\@dblarg{\@sict{#1}{#2}{#3}{#4}{#5}{#6}}}}

\def\@sict#1#2#3#4#5#6[#7]#8{\ifnum #2>\c@secnumdepth
     \def\@svsec{}\else
     \refstepcounter{#1}\edef\@svsec{\csname the#1\endcsname}\fi
     \@tempskipa #5\relax
      \ifdim \@tempskipa>\z@
        \begingroup #6\relax
          \@hangfrom{\hskip #3\relax\@svsec.\hskip 0.1em}
                    {\interlinepenalty \@M #8\par}
        \endgroup
       \csname #1mark\endcsname{#7}\addcontentsline
         {toc}{#1}{\ifnum #2>\c@secnumdepth \else
                      \protect\numberline{\csname the#1\endcsname}\fi
                    #7}\else
        \def\@svsechd{#6\hskip #3\@svsec #8\csname #1mark\endcsname
                      {#7}\addcontentsline
                           {toc}{#1}{\ifnum #2>\c@secnumdepth \else
                             \protect\numberline{\csname the#1\endcsname}\fi
                       #7}}\fi
     \@xsect{#5}}

\def\@sect#1#2#3#4#5#6[#7]#8{\ifnum #2>\c@secnumdepth
     \def\@svsec{}\else
     \refstepcounter{#1}\edef\@svsec{\csname the#1\endcsname\hskip 0.5em }\fi
     \@tempskipa #5\relax
      \ifdim \@tempskipa>\z@
        \begingroup #6\relax
          \@hangfrom{\hskip #3\relax\@svsec}{\interlinepenalty \@M #8\par}
        \endgroup
       \csname #1mark\endcsname{#7}\addcontentsline
         {toc}{#1}{\ifnum #2>\c@secnumdepth \else
                      \protect\numberline{\csname the#1\endcsname}\fi
                    #7}\else
        \def\@svsechd{#6\hskip #3\@svsec #8\csname #1mark\endcsname
                      {#7}\addcontentsline
                           {toc}{#1}{\ifnum #2>\c@secnumdepth \else
                             \protect\numberline{\csname the#1\endcsname}\fi
                       #7}}\fi
     \@xsect{#5}}

\def\thesection {\arabic{section}}
\def\thesubsection {\thesection.\arabic{subsection}}
\renewcommand\section{\@startsiction{section}{1}{\z@}{-0.24in}{0.10in}
             {\large\bf\raggedright}}
\renewcommand\subsection{\@startsection{subsection}{2}{\z@}{-0.20in}{0.08in}
                {\normalsize\bf\raggedright}}
\renewcommand\subsubsection{\@startsection{subsubsection}{3}{\z@}{-0.18in}{0.08in}
                {\normalsize\sc\raggedright}}
\renewcommand\paragraph{\@startsiction{paragraph}{4}{\z@}{1.5ex plus
  0.5ex minus .2ex}{-1em}{\normalsize\bf}}
\renewcommand\subparagraph{\@startsiction{subparagraph}{5}{\z@}{1.5ex plus
  0.5ex minus .2ex}{-1em}{\normalsize\bf}}
\makeatother

%%%%%%%%%%%%%%%%%%%%%%%%%%%%%%%%%%%%%%%%%%%%%%%%%%%%%%%%%%%%%%%%%%%%%%%%%%%
%                               A B S T R A C T
%%%%%%%%%%%%%%%%%%%%%%%%%%%%%%%%%%%%%%%%%%%%%%%%%%%%%%%%%%%%%%%%%%%%%%%%%%%

\renewenvironment{abstract}
{\centerline{\large\bf Abstract}\vspace{0.7ex}%
  \bgroup\leftskip 20pt\rightskip 20pt\small\noindent}%
{\par\egroup\vskip 0.25ex}

%%%%%%%%%%%%%%%%%%%%%%%%%%%%%%%%%%%%%%%%%%%%%%%%%%%%%%%%%%%%%%%%%%%%%%%%%%%
%                         KEYWORDS THEOREM ACK NOTES
%%%%%%%%%%%%%%%%%%%%%%%%%%%%%%%%%%%%%%%%%%%%%%%%%%%%%%%%%%%%%%%%%%%%%%%%%%%

\newenvironment{keywords}
{\bgroup\leftskip 20pt\rightskip 20pt \small\noindent{\bf Keywords:} }%
{\par\egroup\vskip 0.25ex}

\newcommand{\BlackBox}{\rule{1.5ex}{1.5ex}}  % end of proof
\newenvironment{proof}{\par\noindent{\bf Proof\ }}{\hfill\BlackBox\\[2mm]}
\newtheorem{example}{Example}
\newtheorem{theorem}{Theorem}
\newtheorem{lemma}[theorem]{Lemma}
\newtheorem{proposition}[theorem]{Proposition}
\newtheorem{remark}[theorem]{Remark}
\newtheorem{corollary}[theorem]{Corollary}
\newtheorem{definition}[theorem]{Definition}
\newtheorem{conjecture}[theorem]{Conjecture}
\newtheorem{axiom}[theorem]{Axiom}

\long\def\acks#1{\vskip 0.3in\noindent{\large\bf Acknowledgments}\vskip 0.2in\noindent #1}
\long\def\researchnote#1{\noindent {\LARGE\it Research Note} #1}

%%%%%%%%%%%%%%%%%%%%%%%%%%%%%%%%%%%%%%%%%%%%%%%%%%%%%%%%%%%%%%%%%%%%%%%%%%%
%                        FIRST PAGE, TITLE, AUTHOR
%%%%%%%%%%%%%%%%%%%%%%%%%%%%%%%%%%%%%%%%%%%%%%%%%%%%%%%%%%%%%%%%%%%%%%%%%%%

\newlength\aftertitskip     \newlength\beforetitskip
\newlength\interauthorskip  \newlength\aftermaketitskip

%% Changeable parameters.
\setlength\aftertitskip{0.1in plus 0.2in minus 0.2in}
\setlength\beforetitskip{0.05in plus 0.08in minus 0.08in}
\setlength\interauthorskip{0.08in plus 0.1in minus 0.1in}
\setlength\aftermaketitskip{0.3in plus 0.1in minus 0.1in}

%=================================================================
%%CONTENTS
%=================================================================

% Definitions of handy macros
\newcommand{\fracpartial}[2]{\frac{\partial #1}{\partial  #2}}

\begin{document}
\begin{CJK*}{GBK}{song}

\title{Data Mining Project: Prediction of Tokyo Water Supply}
\fancyhf{}
\fancyhead[CE]{Roi ZHAO}
\fancyfoot[RO,LE]{\thepage}
\fancyfoot[LO]{NEC Labs China}
\fancyheadoffset{4cm}

%\maketitle
{ \huge \bfseries \centering 东京都水道局供水预测项目技术报告}\\[0.4cm]
{\color{blue}赵仁豫},zhao\_renyu{@}nec{.}cn \emph{2013/10/15}\\
\hrule
\vspace{0.2cm}
\thispagestyle{plain}

\begin{abstract}%   <- trailing '%' for backward compatibility of .sty file

This paper describes the mixtures-of-trees model, a probabilistic
model for discrete multidimensional domains.  Mixtures-of-trees
generalize the probabilistic trees of
in a different and complementary direction to that of Bayesian networks.
We present efficient algorithms for learning mixtures-of-trees
models in maximum likelihood and Bayesian frameworks.
We also discuss additional efficiencies that can be
obtained when data are ``sparse,'' and we present data
structures and algorithms that exploit such sparseness.
Experimental results demonstrate the performance of the
model for both density estimation and classification.
We also discuss the sense in which tree-based classifiers
perform an implicit form of feature selection, and demonstrate
a resulting insensitivity to irrelevant attributes.

\end{abstract}

\begin{keywords}
  Bayesian Networks, Mixture Models,
\end{keywords}

\section{项目跟踪}
\CJKindent
从9月20日开始,项目进入最后阶段,主要是对细节进行调整,统计数据报告。
\subsection{9月20日}
9月20日当日确定了后期分析的目标计划,共有三个方面,其中两个是对总供水量和各点供水量的预测精度进行可视化,另外一个是提高各点供水量预测精度。其中提高预测精度的方案依旧是变更学习数据区间,从两年所有数据缩减到东京震灾后的区间。实施过程中至10月8日一直在进行数据可视化工作(10月8日至今尚未看到新的日报,但邮件内容显示有在分析各点供水量的周期性)。\\
9月24日到9月27日为上述计划实施的第一周,做成的可视化图形有:\\
-误差率对比图(现行系统-异种混合)\\
\centerline{\includegraphics[width=15cm]{S01.jpg}}\\
-误差率散布图(异种混合相对线性系统的改善量)\\
\centerline{\includegraphics[width=15cm]{S02.jpg}}\\
-各点供水量与误差率散布图\\
\centerline{\includegraphics[width=15cm]{S03.jpg}}\\
-天气状况变化统计图(晴云雨雪Markov序列)\\
\centerline{\includegraphics[width=15cm]{S04.jpg}}\\
-天气状况变化与误差率散布图\\
\centerline{\includegraphics[width=15cm]{S05.jpg}}\\

10月1日到10月4日第二周,做成的可视化图形有:\\
-各点误差率与供水量关系分布图\\
\centerline{\includegraphics[width=15cm]{S06.jpg}}\\
-气温差分与误差率散点图\\
\centerline{\includegraphics[width=15cm]{S07.jpg}}\\
-对比Tuning3d、3e的误差绝对值数据做成表格\\

10月7日到10月8日主要对之前数据做了修正:\\
-删除误差率比较图中数据不全的点\\
-修改坐标轴描述,变量表示颜色等\\



\section{总结}
按照中央研究院提供的标准作业流程(Standard Business Process),L1层(大纲)共有16个阶段:
\begin{enumerate}
\setlength{\itemsep}{0pt}
  \item Sales \& Interviews
  \item 1st proposal
  \item Preparation for a trial
  \item Trial analysis
  \item Trial reports
  \item Strategies of intellectual property and commercialization
  \item Proposing the system
  \item Definitive contract
  \item Launching a project
  \item Regular analysis
  \item Requirement definition
  \item Design
  \item Production
  \item Test
  \item System installation
  \item Operation
\end{enumerate}
从8月28日我们作为NECLC加入东京都水道局供水预测项目(水道局项目)到目前为止,
该项目经历了试验准备(3, Preparation for a trial),试验分析(4,Trial analysis)两个阶段,并将于下周五(2013/10/25)进入试验结果报告(5,Trial reports)阶段。\\
本项目的主要目标是利用东京都水道局(东京地方公共自来水供水单位)的供水量历史数据,预测东京都东部23区、西部26市町村各点(66个/97个)及总供水需求量,并借助更加准确的预测,达到节省水利输送电力消耗,节约成本的目的。

\subsection{项目管理}
该项目是NEC中央研究院主导的项目,包括BU(Business Unit)和NES(Software)。以下介绍项目相关人员、周期及其他相关信息。

\subsubsection{人员安排}
参与本项目的成员包括来自NECCRL(中央研究院)的青木 健児, 千葉 雄樹(Cloud System 研究所), 本橋 洋介,以及来自NECSoftware的瀬戸 美星(PF System), 保坂 真奈美, 小泉 美帆(NAS,后期加入)和我们(中国研究院)。
其中濑户和保坂负责系统运行,对数据运用策略得到分析结果,他们向青木报告。千叶在9月20 日前参与策略讨论指导,其后从邮件看未见参与项目活动。\\

\subsubsection{项目日程}
在我们参与的项目周期内,项目进展包括三个方面:预测变量调节(説明変数チューニング);数据更新;以及结果可视化。其中数据更新有两次,一次是加和各小时数据获得日总量数据,另一次是从客户获得天气数据(及客户提供的预测天气数据);结果可视化工作主要在项目后期大量展开;
数据更新是项目进展中最重要的一个方面,实质性提升预测结果的有3个重要变化:说明变量加入前一周平均供水量、学习区间调整缩小范围、以及更新天气数据,具体情况将在下面“数据技术”部分讨论。\\
这一周期可以分为两个阶段,第一阶段项目预测工作使用“现行系统”,第二阶段使用异构混合技术,大致以9月3日为间隔。按照项目进度,也可以按照工作重点从9月20日分为两个阶段,前一阶段对数据依照预先设置的目标计划的进行大量分析试验,参数检查,且依据每日试验结果追加新的目标设定,后一个阶段根据整个前一阶段的结果作出相对规模较小的改进,并进行结果可视化工作。

\subsubsection{项目文档}
前期准备文档包括前一阶段向客户提交的Proposal、分析任务目标设定、关键概念等。\\
所有文档存放于局域网ftp,地址及命名规则大致形如\verb|\\10.56.41.236\pp\03_探索案件\02. 水道局|,该目录下分析有★取扱注意★現行システムとの比較検証、分析資料、result等文件夹。\\
受权限限制,只能看到青木拷贝至服务器相关neclc文件夹内(SAMPO2服务器 \verb|/mnt/M100_SAMPO2_U_00/project/tokyo_waterworks/neclc/|)的文件,其中9月3 日前只有除前期准备文档外只有两份WeeklyReport的pptx文件,而从9月2日开始每个工作日有格式如\verb|daily_report\daily_report_20131017.xlsx|的日报,以表格形式标记工作进度、追加任务及说明变量调整。自9月9日起每个工作日有格式如\verb|予測精度比較_20131007.pptx| 的预测结果比较文档,用于展示每次调整后异构混合方法与之前所得数据的横向、纵向结果比较。

\subsection{数据技术}
本项目至9月3日前采用的是现行预测系统(現行予測システムと),之后采用NEC研究院的异构混合技术,以现行系统作为比较基准。另外在前一阶段向用户提交Proposal及9月3 日前的分析中包括每小时水量预测,此后不再涉及,均为各点及东京都供水日总量预测。项目中供水量数据仅有东京都日总水量,以及各目标预测点每小时供水量(加和与总量不等)。\\
异构混合方法(FAB HLM)最初预测使用的说明变量(feature)包括年月日类别、星期类别、假日类型、1 天前、1 周前、2 周前供水量数据、天气,(在每日19:00)预测当日22:00开始的一天、或翌日22:00开始的一天总供水量(各点及总量)。预测区间为2012/4/13至2012/9/30,初始学习区间为2010/1/1至2012/4/12。以下说明每次调整:
\begin{itemize}
  \item 天气数据从每小时数值,调整为每6小时为单位的统计值(预处理)
  \item Tuning1 说明变量加入2日前、3日前水量,去除年份(9/10)
  \item Tuning2 对DataSet中日分类做不明变动,因结果差后被取消(9/11)
  \item Tuning3a 加入从客户处获得的大河内天气数据(9/12)
  \item Tuning3b 修正3a中变量名(汗。)
  \item Tuning3c 加入预测日前7日总供水量平均作为说明变量 (9/13)
  \item Tuning3d 算法学习数据区间从两年缩减到东京震灾后2011/3-2012/2 (9/14)
  \item Tuning3e 加入从客户处获得的“预测”天气作为说明变量(9/19)
\end{itemize}

\subsubsection{算法理论}
本项目属于单目标时间序列回归预测分析,中央研究院采用的FABHLM技术本质上是层次线性模型,使用改进的渐进贝叶斯方法计算出模型参数。这一模型的优点在于适用范围广,较其他混合模型而言结构更简单,适用于周期性较强的时间序列分析。其他可以采用的模型还有样条分析等。FAB算法的优点在于收敛速度快,较普通EM算法一致性好(理解为FL 散度更小),这里当然也可以采用变分贝叶斯算法。

\subsubsection{项目背景}
这一项目的目标客户,东京都水道局是日本的地方公营企业,负责东京都23区及除少数地区外的多摩26市町供水。供水水泵耗电量非常之大(2009年中国供水耗电总量为108.3亿千瓦时,按0.5 元每千瓦时算合人民币50多亿元),而提高供水需求预测精度可以降低无效或折返供水耗电,从而节约成本,因此本项目的目标是尽可能提高东京地方供水预测水平。

\subsubsection{数据特征}
有关数据各变量内容前文已经述及,
下面对数据特征涉及到的一些概念简单解释:
\begin{description}
  \item[DataSet] 指数据库中的数据,即预测工作第一步在Postgre数据库中整合观测变量及日期属性等得到的数据集。
  \item[投入值] 在现行系统中,某时刻根据当前状况修正的说明变量值,加入后重新计算预测函数,更新预测结果。
  \item[予測先] 预测当时(区别于预测前1天)。
  \item[门函数] HLM模型中每一层类比于决策树判断选择何种划分的函数。
\end{description}

\subsubsection{问题难点}
本项目的难点在于需要得到的预测精度非常高,例如作为异构混合技术比较对象的“现行系统”的结果如下表:\\
    \begin{center}
    \begin{tabular}{lrr}
        预测区间 & 预测値 & 修正预测值\\
        \hline
        2007/1/1~2012/9/30 & 2.10\% & 1.55\%\\
        2010/1/1~2012/9/30 & 2.20\% & 1.48\%\\
        2011/3/11~2012/9/30 & 2.12\% & 1.40\%\\
    \end{tabular}\\
    \end{center}
误差率已经非常小,因而取得明显优势的改进显得非常困难。\\
同时项目中对于我们作为合作观察者而言,存在文件不足和交流不畅的问题,举例如下:
\begin{itemize}
  \item 对于“现行系统”,我们一直不知道指的是水道局目前使用的系统,
  还是NEC研究院采用先前的系统;也不知道它使用何种算法。邮件沟通并未解决此问题。(【Mail Ref】Re: questions about water demand @Tue 9/17/2013 3:50 PM by Mihoshi Seto)
  \item 对于“投入值”(投入値)这个概念,主要因为对项目实际情境不了解(客户操作背景),多次讨论后才逐渐清楚。(【Mail Ref】 RE: about water demand prediction PJ @Fri 9/6/2013 1:59 PM by Kenji Aoki)
\end{itemize}

通过参与项目,我们也获得很多经验:\\\-首先是要对所使用的模型和算法有明确的认识,才能更好地选择说明变量(feature),以本项目为例,变量的线性组合虽然在普通线性模型中是无效变量(非基变量),但是在混合模型中,变量的线性组合可能成为比变量本身更好的说明变量;\\\-另外要根据项目周期合理分配时间,有计划的设计调试说明变量效果的顺序步骤,提前考虑到过程中可能追加的分析目标,留出合理的时间块;\\\- 对于要呈现给客户的案例来说,得出合适的预测结果远远不够,还需要大量的可视化工作,将结果以最直观整洁的形式展现出来;\\\- 最后是项目文档对概念定义需要完善,尤其对缩写和易混淆概念的解释,对新加入和参考学习的人理解项目非常重要。

\subsection{其他}
\subsubsection{邮件格式}
主题标明所涉项目及所讨论问题提出的日期或版本号,可以很容易辨认分类。

\noindent\framebox[1.5\height][r]{$\pi$}\framebox[2.5\height][r]{$\pi$}\framebox[4.5\height][r]{$\pi$}
\vskip 0.2in

\end{CJK*}
\end{document}
